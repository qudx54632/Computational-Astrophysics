\documentclass{article}
\usepackage{amsmath}
\usepackage[a4paper, margin=2cm]{geometry}

\title{Burger's Equation}
\begin{document}
\maketitle
\section{MC (Monotonized Central) Limiter}
The MC limiter is a slope limiter used in finite-volume and finite-difference schemes to achieve second-order accuracy in smooth regions while avoiding spurious oscillations near discontinuities (shocks).

The idea is to construct piecewise linear profiles inside each cell but limit their slopes to maintain monotonicity.

The formula here we use: For cell values $a_{i-1}$, $a_i$, $a_{i+1}$, we define
\begin{equation}
\Delta^- = a_i - a_{i-1}, \quad
\Delta^+ = a_{i+1} - a_i, \quad
\Delta^c = \frac{1}{2}(a_{i+1} - a_{i-1}).
\end{equation}

Then the limited slope is
\begin{equation}
\sigma_i =
\begin{cases}
\text{sign}(\Delta^c) \, \min\!\Big( |\Delta^c|, 2|\Delta^-|, 2|\Delta^+| \Big), &\quad \Delta^- \Delta^+ > 0, \\
0, &\quad \text{otherwise}.
\end{cases}	
\end{equation}

Properties:
\begin{itemize}
	\item Second-order accurate in smooth regions.
	\item Non-oscillatory near discontinuities.
	\item Less diffusive than minmod, more robust than unlimited central difference.
	\item Usage: Common in MUSCL--Hancock and Godunov-type schemes for hyperbolic conservation laws.
\end{itemize}

\section{Predictor Step and Interface Reconstruction}
The ``slope'' we compute is the limited cell-edge jump per $\Delta x$, i.e.
\begin{equation}
	\left(\frac{\partial a}{\partial x}\right)_i \, \Delta x.
\end{equation}
This slope enters directly into the upwinded linear reconstructions at the cell interfaces. To achieve second-order accuracy in both space and time, the MUSCL--Hancock scheme evolves the piecewise linear states forward by half a timestep. 

For Burgers' equation, where the characteristic speed is $a = u_i$, the interface states become
\begin{equation}
u_{i+1/2}^- = u_i - \tfrac{1}{2}\!\left(1 + \frac{u_i \,\Delta t}{\Delta x}\right)\sigma_i,
\end{equation}
\begin{equation}
u_{i-1/2}^+ = u_i + \tfrac{1}{2}\!\left(1 - \frac{u_i \,\Delta t}{\Delta x}\right)\sigma_i.
\end{equation}

Here, the factors $\left(1 \pm \tfrac{u_i \Delta t}{\Delta x}\right)$ arise from the characteristic evolution over $\Delta t/2$. Without these correction terms, the method would be only second-order in space but first-order in time. Thus, the MUSCL--Hancock predictor step ensures overall second-order accuracy while preserving stability near discontinuities.

\section{Weighted Essentially Non-Oscillatory (WENO) Reconstruction}

The WENO (Weighted Essentially Non-Oscillatory) method is a high-order accurate spatial reconstruction scheme designed for solving hyperbolic conservation laws, especially those involving shocks or sharp gradients. It achieves high-order accuracy in smooth regions and avoids spurious oscillations near discontinuities by adaptively weighting multiple stencils.

Let $q_i$ be the cell-averaged quantity. To compute the left-biased value $q_{i+1/2}^-$ at the interface $x_{i+1/2}$, WENO uses $r$ overlapping stencils of $r$ points each:
\[
S_k = \{ q_{i - r + k}, \dots, q_{i + k - 1} \}, \quad k = 1, \dots, r.
\]
Each stencil $S_k$ yields a candidate polynomial approximation evaluated at the interface $x_{i+1/2}$:
\[
q^{(k)}_{i+1/2} = \sum_{\ell=1}^r a_{k\ell} q_{i + k - \ell},
\]
where the coefficients $a_{k\ell}$ are precomputed for each stencil and ensure that each $q^{(k)}_{i+1/2}$ is a $(r-1)$-degree polynomial reconstruction with order $r$ accuracy over stencil $S_k$.

These polynomials form the building blocks for the final nonlinear reconstruction:
\[
q_{i+1/2}^- = \sum_{k=1}^r \omega_k q^{(k)}_{i+1/2}.
\] 
The final WENO reconstruction is a convex combination:
\[
q_{i+1/2}^- = \sum_{k=1}^r \omega_k q^{(k)}_{i+1/2},
\]
where $\omega_k$ are nonlinear weights that favor smooth stencils and reduce to linear weights $C_k$ in smooth regions.

\section*{Smoothness Indicators and Weights}

The smoothness indicators $\beta_k$ measure the smoothness of $q$ over stencil $S_k$, typically using:
\[
\beta_k = \sum_{l=1}^r \sum_{m=1}^l \sigma_{k\ell m} q_{i+k-\ell} q_{i+k-m},
\]
with precomputed constants $\sigma_{klm}$.

Then, the nonlinear weights are:
\[
\alpha_k = \frac{C_k}{\varepsilon + \beta_k^2}, \quad
\omega_k = \frac{\alpha_k}{\sum_j \alpha_j},
\]
where $\varepsilon \ll 1$ prevents division by zero and $C_k$ are the optimal linear weights for the desired order of accuracy.

\section*{Examples of WENO Coefficients}

Below we list the coefficients used in the WENO reconstruction for two representative orders.

\paragraph{Example 1: \( r = 2 \)} 
This uses 3-point stencils (2 on each side of an interface):

\[
C^{(2)} = \left[ \frac{1}{3},\ \frac{2}{3} \right],
\]
\[
a^{(2)} = \frac{1}{2}
\begin{bmatrix}
3 & -1 \\
1 & 1
\end{bmatrix},
\]
\[
\sigma^{(2)} = 
\left\{
\begin{array}{ll}
\sigma_1 =
\begin{bmatrix}
1 & 0 \\
0 & 0
\end{bmatrix},
& 
\sigma_2 =
\begin{bmatrix}
1 & 0 \\
-4 & 4
\end{bmatrix}
\end{array}
\right.
\]

\paragraph{Example 2: \( r = 3 \)} 
This uses 5-point stencils:

\[
C^{(3)} = \left[ \frac{1}{10},\ \frac{6}{10},\ \frac{3}{10} \right],
\]
\[
a^{(3)} = \frac{1}{6}
\begin{bmatrix}
2 & -7 & 11 \\
-1 & 5 & 2 \\
2 & 5 & -1
\end{bmatrix},
\]
\[
\sigma^{(3)} = 
\left\{
\begin{array}{ll}
\sigma_1 =
\begin{bmatrix}
1 & 0 & 0 \\
-4 & 4 & 0 \\
3 & -6 & 3
\end{bmatrix}, &

\sigma_2 =
\begin{bmatrix}
1 & 0 & 0 \\
-4 & 4 & 0 \\
3 & -6 & 3
\end{bmatrix}, \\[1em]

\sigma_3 =
\begin{bmatrix}
1 & 0 & 0 \\
-4 & 4 & 0 \\
3 & -6 & 3
\end{bmatrix}
\end{array}
\right.
\]

The smoothness indicators \( \beta_k \) are computed via:
\[
\beta_k = \sum_{\ell=1}^r \sum_{m=1}^\ell \sigma_{k\ell m} q_{i+k-\ell} q_{i+k-m},
\]
using the above \( \sigma \) coefficients.



\section*{WENO-M Mapping (Optional)}

The mapped WENO (WENO-M) improves resolution near critical points by remapping the nonlinear weights $\omega_k$ through a transformation:
\[
\alpha_k^{\text{map}} = \frac{\omega_k (C_k + C_k^2 - 3 C_k \omega_k + \omega_k^2)}{C_k^2 + \omega_k(1 - 2 C_k)},
\]
and normalizing again to obtain final weights.

\section*{Time Integration and Flux Splitting}

To apply WENO to nonlinear problems (e.g. Burgers' equation), a flux splitting is used:
\[
f^\pm(u) = \frac{1}{2} (f(u) \pm \alpha u),
\]
with $\alpha = \max |f'(u)|$ for global Lax-Friedrichs splitting.

Each $f^\pm$ is reconstructed with left- or right-biased WENO. The resulting interface flux is:
\[
f_{i+1/2} = f^+_{i+1/2} + f^-_{i+1/2}.
\]

The spatial derivative $\partial_x f$ is then approximated by a finite difference of the numerical flux, and advanced in time using high-order Runge-Kutta (e.g. RK4).

\section*{Convergence}

The WENO-$r$ scheme formally achieves $(2r-1)$th order accuracy in smooth regions. For example:
\begin{itemize}
    \item $r=3$ yields 5th-order accuracy,
    \item $r=4$ yields 7th-order accuracy,
    \item $r=5$ yields 9th-order accuracy.
\end{itemize}

This is confirmed by convergence tests against smooth solutions, such as Gaussian profiles evolved by the method of characteristics.


\end{document}