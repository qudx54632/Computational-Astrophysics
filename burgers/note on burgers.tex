\documentclass{article}
\usepackage{amsmath}

\begin{document}
\section{MC (Monotonized Central) Limiter}
The MC limiter is a slope limiter used in finite-volume and finite-difference schemes to achieve second-order accuracy in smooth regions while avoiding spurious oscillations near discontinuities (shocks).

The idea is to construct piecewise linear profiles inside each cell but limit their slopes to maintain monotonicity.

The formula here we use: For cell values $a_{i-1}$, $a_i$, $a_{i+1}$, we define
\begin{equation}
\Delta^- = a_i - a_{i-1}, \quad
\Delta^+ = a_{i+1} - a_i, \quad
\Delta^c = \frac{1}{2}(a_{i+1} - a_{i-1}).
\end{equation}

Then the limited slope is
\begin{equation}
\sigma_i =
\begin{cases}
\text{sign}(\Delta^c) \, \min\!\Big( |\Delta^c|, 2|\Delta^-|, 2|\Delta^+| \Big), &\quad \Delta^- \Delta^+ > 0, \\
0, &\quad \text{otherwise}.
\end{cases}	
\end{equation}

Properties:
\begin{itemize}
	\item Second-order accurate in smooth regions.
	\item Non-oscillatory near discontinuities.
	\item Less diffusive than minmod, more robust than unlimited central difference.
	\item Usage: Common in MUSCL--Hancock and Godunov-type schemes for hyperbolic conservation laws.
\end{itemize}

\section{Predictor Step and Interface Reconstruction}
The ``slope'' we compute is the limited cell-edge jump per $\Delta x$, i.e.
\begin{equation}
	\left(\frac{\partial a}{\partial x}\right)_i \, \Delta x.
\end{equation}
This slope enters directly into the upwinded linear reconstructions at the cell interfaces. To achieve second-order accuracy in both space and time, the MUSCL--Hancock scheme evolves the piecewise linear states forward by half a timestep. 

For Burgers' equation, where the characteristic speed is $a = u_i$, the interface states become
\begin{equation}
u_{i+1/2}^- = u_i - \tfrac{1}{2}\!\left(1 + \frac{u_i \,\Delta t}{\Delta x}\right)\sigma_i,
\end{equation}
\begin{equation}
u_{i-1/2}^+ = u_i + \tfrac{1}{2}\!\left(1 - \frac{u_i \,\Delta t}{\Delta x}\right)\sigma_i.
\end{equation}

Here, the factors $\left(1 \pm \tfrac{u_i \Delta t}{\Delta x}\right)$ arise from the characteristic evolution over $\Delta t/2$. Without these correction terms, the method would be only second-order in space but first-order in time. Thus, the MUSCL--Hancock predictor step ensures overall second-order accuracy while preserving stability near discontinuities.


\end{document}